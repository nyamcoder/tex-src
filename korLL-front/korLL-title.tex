%%% korLL-title.tex :: mito / nyamcoder %%%
\documentclass{article}
\usepackage{kotex}
\usepackage{graphicx} 
\newsavebox\IBox
	\sbox\IBox{\includegraphics[scale=.1835]{pic/Korean-cuisine-Banchan-02_32ooX24oo-oil-rand32-spiegel2.eps}}
\usepackage[svgnames]{xcolor}
\usepackage{pstricks,pst-grad,pst-text,pst-blur}
\usepackage{chancery}	% emulating Zapf Chancery font 
\usepackage{soul}		% letter spacing (\sodef)
\pagestyle{empty}
\begin{document}
\vspace*{-1cm}
\begin{centering}
%\textsf{\LARGE Michael\enskip Topp}\\[5mm]
\end{centering}
%\psset{linewidth=0.5pt,blur=true,blurradius=0.1cm,shadow=true,linecolor=black!60,shadowcolor=red!20,shadowsize=0.3cm,fillstyle=gradient,gradbegin=darkgray,gradend=white,cmyk}
% \rput(10,1.5){\pscharpath{\fontsize{60}{60}\selectfont 문자와\enskip 표기}}	 
\SetHangulFonts{utbm}{utgt}{uttz} % ok: utpga,utbm,utdn,utjsr 
	% black teal brown olive  
\hspace*{-2cm}%\vspace*{-5cm}
\begin{pspicture}[shift=-2cm](15,20)
\rput[lb](-.3,-.3){\usebox\IBox}
%\psset{linewidth=0.5pt,blur=true,blurradius=0.1cm,shadow=true,linecolor=black!60,shadowcolor=black!20,shadowsize=0.3cm}
\psframe[linecolor=SeaGreen](15,20.15) % linecolor=AliceBlue  blurbg=MintCream linewidth=.5pt,linecolor=HotPink 
\rput(7.5,18.7){\textsf{\LARGE \textcolor{LightYellow}{Michael\enskip Topp}}}
\psset{linestyle=none,linewidth=.01,linecolor=HotPink,
%blur=true,blurradius=0.07cm,blurbg=Honeydew,
%shadow=false,shadowcolor=SeaGreen,shadowsize=0.3cm,shadowangle=-70,
fillstyle=gradient,gradbegin=DeepSkyBlue,gradend=LightCyan,gradangle=-40,cmyk}
	\rput(7.5,13.5){\pscharpath{\fontsize{60}{60}\selectfont\shortstack{Koreanisch-\qquad\quad\\\qquad\quad Rezepte\\[-.15em]\fontsize{30}{30}\selectfont für}}}

\psset{linestyle=none,linewidth=.01,%linecolor=black!60,
blur=true,blurradius=0.25cm,blurbg=LightYellow,%
shadow=true,shadowcolor=SeaGreen,shadowsize=0.2cm,shadowangle=-70,%
fillstyle=gradient,gradangle=90,gradbegin=cyan,gradend=red,cmyk}
	\rput(7.5,10.3){\pscharpath{\fontsize{60}{60}\selectfont Linux \&\,\LaTeX}}

\psset{linestyle=none,linewidth=.01,%
blur=true,blurradius=0.15cm,%
shadow=true,shadowsize=0.15cm,%
fillstyle=gradient,gradbegin=cyan,gradend=red,gradmidpoint=1,cmyk}% ,gradmidpoint=.65 
	\rput(7.5,7){\pscharpath{\rmfamily\fontsize{30}{30}\selectfont\itshape{\qquad리눅스{\fontsize{20}{20}\selectfont와}\enskip라텍\qquad\qquad}}}
%	\pscustom[gradbegin=Teal,gradend=LightBlue]

\psset{linewidth=0.5pt,
blur=false,%blurradius=0.1cm,
shadow=false,%linecolor=black!60,shadowcolor=red!20,shadowsize=0.2cm,  gradmidpoint=1.5, 
fillstyle=gradient,gradbegin=LightCyan,gradend=DeepSkyBlue,gradangle=-40,cmyk}
	\rput(7.5,6.17){\pscharpath{\shortstack{\fontsize{20}{20}\selectfont\qquad\qquad\qquad\itshape{\qquad\qquad을\enskip위한}\\[4mm]\fontsize{30}{30}\selectfont\itshape{한국식\enskip조리법}}}}

\sodef\cs{}{-.05em}{.35em}{.7em}
\psset{linewidth=0.01pt,linecolor=LightYellow,
blur=false,%blurradius=0.1cm,
fillstyle=solid,fillcolor=SeaGreen!80,
shadow=false%,shadowcolor=SeaGreen,shadowsize=0.3cm
}
	\rput(7.5,3.5){\pscharpath{\sffamily\fontsize{30}{30}\selectfont\cs{Korean Cookbook}}}
	\rput(7.5,2.5){\pscharpath{\sffamily\fontsize{30}{30}\selectfont\cs{for Linux \& \LaTeX}}}
%\psgrid[gridcolor=orange](16,26)

\end{pspicture}
\pagebreak\\
\pagebreak %%%%%%  schmutztitel / mock title %%%%%%
%\vspace*{-2cm}
\hspace*{-1.5cm}
\raisebox{-2cm}[20cm]{
\begin{pspicture}[shift=3cm](15,20)
\psframe(15,20.15)
\rput(7.5,18.7){\textsf{\LARGE Michael\enskip Topp}}
\rput(7.5,13.5){\pscharpath{\fontsize{60}{60}\selectfont\shortstack{Koreanisch-\qquad\quad\\\qquad\quad Rezepte\\[-.15em]\fontsize{30}{30}\selectfont für}}}
\rput(7.5,10.3){\pscharpath{\fontsize{60}{60}\selectfont Linux \&\,\LaTeX}}
\rput(7.5,7){\pscharpath{\rmfamily\fontsize{30}{30}\selectfont\itshape{\qquad리눅스{\fontsize{20}{20}\selectfont와}\enskip라텍\qquad\qquad}}}
\rput(7.5,6.17){\pscharpath{\shortstack{\fontsize{20}{20}\selectfont\qquad\qquad\qquad\itshape{\qquad\quad\enskip\,을\enskip위한}\\[4mm]\fontsize{30}{30}\selectfont\itshape{한국식\enskip조리법}}}}
\sodef\cs{}{-.05em}{.35em}{.7em}
\rput(7.5,3.5){\pscharpath{\sffamily\fontsize{30}{30}\selectfont\cs{Korean Cookbook}}}
\rput(7.5,2.5){\pscharpath{\sffamily\fontsize{30}{30}\selectfont\cs{for Linux \& \LaTeX}}}
\rput(7.5,-1){\sffamily\fontsize{16}{16}Version 0.1.2}
\end{pspicture}}
%\pagebreak

\end{document}


* compilation

$ latex korLL-title.tex
$ dvips -o korLL-title.ps korLL-title.dvi
$ ps2pdf korLL-title.ps

+++ note: in ko.TeX 2013, utbm (Un Bom) seems not to work as usual and is replaced by a dummy Myeongjo font; 


* image resource: https://en.wikipedia.org/wiki/File:Korean.cuisine-Banchan-02.jpg


