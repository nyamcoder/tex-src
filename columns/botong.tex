\documentclass[11pt]{article}		%%% botong.tex :: mito / nyamcoder  %%%
\usepackage{CJKutf8}
\usepackage{multicol}
\usepackage{geometry}
\geometry{paperwidth=21cm,paperheight=29.7cm,margin=3cm}
\usepackage[cam,b4,center]{crop}
\setlength\columnseprule{.4pt}
\newcommand\deu{\LaTeX{} ist ein Satzsystem, das für viele Arten von Schriftstücken verwendet werden kann, von einfachen Briefen bis zu kompletten Büchern. Besonders geeignet ist es für wissenschaftliche oder technische Dokumente. \LaTeX{} ist für praktisch alle verbreiteten Betriebssysteme verfügbar.}
\newcommand\kor{\LaTeX은 과학 및 수학 문서를 작성하는데 적당한 조판 시스템으로서 대단히 뛰어난 조판 품질을 얻을 수 있게 한다. 또한 단순한 편지에서 완전한 단행 본에 이르기까지 여러 종류의 문서를 만드는 데도 적합하다. \LaTeX은 PC나 매킨토시에서 대규모 UNIX나 VMS에 이르기까지 대부분의 컴퓨터에서 실행된다.} 
%\setlength\parindent{0em}

\begin{document}
\begin{CJK}{UTF8}{mj}
\section*{1-spaltiger  Text im Blocksatz/양끝맞추고 한 단으로 된 텍스트}
\deu\par 

\kor
\section*{2-spaltiger Text in separaten Absatzboxen/한 상자 안에 넣은 두 단으로 된 텍스트}  
\parbox[t]{0.45\linewidth}{\deu}
\hfill
\parbox[t]{0.45\linewidth}{\kor}
\begin{multicols}{3}[\section*{3-spaltiger Text/세 단으로 된 텍스트}]
\deu\\

\kor
\end{multicols}
\setlength\parskip{3em}
\frenchspacing	% taking out extra space after full stops (as common for non-English texts)
\twocolumn
\section*{2-spaltiger Text/두 단으로 된 텍스트}
\deu\\

\kor\begin{quotation}\kor\end{quotation}  
\begin{quote}\deu\end{quote} \begin{verse}\kor\end{verse} 
\onecolumn 
\section*{Und wieder 1-spaltiger Text/그리고 또다시 한 단으로 된 텍스트}  
\linespread{1.5}\selectfont
\deu\\[5.7cm]

\begin{quotation}\kor\end{quotation}

\end{CJK}
\end{document}


* document for testing (German and Korean) text set in one or more justified columns with different sorts of paragraphs and indents,
  such as quotation and verse, and also line widths and (non-)frenchspacing
   o  note #1: since there is no German localisation loaded, German hyphenation does not work and some text lines extend beyond their
      normal column widths (which results into interesting effects)
   o  note #2: by default CJKutf8 turns bold Hangul into a kind of stretched ''poor man's bold face''
   o  note #3: there is an extra package for enabling crop marks 


* paragraph text sources:
   o  /usr/share/texlive/2008/texmf-doc/doc/german/lshort-german/l2kurz.pdf 
   o  /usr/share/texlive/2008/texmf-doc/doc/korean/lshort-korean/lshort-kr.pdf 


* compilation:

$ pdflatex botong.tex


% unterschied \\ --> \par 
%%% \par ignoriert folgende Leerzeilen, kein [Xem] möglich, in benutzerdefinierten Befehlen nicht benutzbar 
% \\ notwendig, da sonst nur " " gesetzt
% bei einem \par Einzug der ersten Zeile voreingestellt; ggf also 
% auch eine Leerzeile im QT bewirkt nur einen einfachen Zeilenumbruch 
% wird nach einem \\{[2em]} noch eine Leerzeile eingefügt, interpretiert LaTeX dies als neuen Abschnitt (paragraph), der entsprechend der Vorgabe eingezogen wird 
% s.a. lb2-p34: indentfirst.sty 
% \noindent nur für die nächste Zeile   
% in par-boxen kein Seitenumbruch erlaubt, weswegen dann ggf auf die nächste Seite verschoben 
% sa multicol.sty -- lb2-193ff
% parbox+minipage-höhe lb2-897ff 
% Durchschuss = Zeilenvorschub lb2-114f,383,205,426 
%%% lb2-114f: spacing.sty 
%%% statt direkt \baselineskip besser \linespread{zahl}\selectfont 
