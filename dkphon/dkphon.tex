\documentclass[a4paper,11pt]{article}		%% dkphon.tex :: mito / nyamcoder
%\usepackage[german]{babel}	% mit german.sty fehler bei ipa! 
\usepackage{CJKutf8}	% für dt sonderzeichen
\pagestyle{empty}
\usepackage{tipa}% [extra,safe] 
%\usepackage{german}	% fehler auch mit [safe]

\setlength{\hoffset}{-2cm}
\setlength{\textwidth}{17cm}
\begin{document}
\frenchspacing % nicht nötig mit babel 
\setlength{\parindent}{0pt}

\begin{CJK}{UTF8}{mj}	% 
\vspace*{-2cm}
\section{Phonetische Betrachtungen: Deutsch -- Koreanisch}
Dieses Kapitel untersucht die phonetischen Eigenheiten des Sprachenpaars Deutsch und Koreanisch. Einleitend dazu ein veranschaulichender Vergleich von Umschrift-Textabschnitten mit den Zeichen des IPA-Systems:\footnote{mit den Beispielen von Klaus Kohler und Hyun Bok Lee aus: Handbook of the International Phonetic Association, Cambridge, Großbritannien, 1999, S. 88f u.\,123}
% bsp: \bibentry{IPA:hb-1999}

\subsection{Deutsch}

\subsubsection{Orthografische Version}
Einst stritten sich Nordwind und Sonne, wer von ihnen beiden wohl der Stärkere wäre, als ein Wanderer, der in einen warmen Mantel gehüllt war, des Weges daherkam. Sie wurden einig, daß derjenige für den Stärkeren gelten sollte, der den Wanderer zwingen würde, seinen Mantel abzunehmen. Der Nordwind blies mit aller Macht, aber je mehr er blies, desto fester hüllte sich der Wanderer in seinen Mantel ein. Endlich gab der Nordwind den Kampf auf. Nun erwärmte die Sonne die Luft mit ihren freundlichen Strahlen, und schon nach wenigen Augenblicken zog der Wanderer seinen Mantel aus. Da mußte der Nordwind zugeben, daß die Sonne von beiden der Stärkere war.

\subsubsection{Enge Transkription der Aufnahme}
\begin{IPA}
a\i ns "StK\i tn z\i\c{c} "nO5tv\i nt Un "zOn@, ve5 f@n im "ba\i dn vol d5 "StE5k@K@ veK@, als a\i n "vand@K5, dE5 \i n a\i n "va5m "mantl g@""hYlt va5, d@s "veg@s da"he5ka:m. z\i{} vU5dn "a\i n\i \c{c}, das "de5jen\i g@ fY5 d@n "StE5k@K@n ""gEltn zOlt@, dE5 d@n "vand@K5 "tsv\i NN vY5d@, za\i m "mantl "abtsU""nemm. dE5 "nO5tv\i m "blis m\i t "al5 "maXt, ab5 je "me5 E5 "blis, dEsto "fEst5 "hYlt@ z\i \c{c} d5 "vand@K5 \i n za\i m "mantl a\i n. "Entl\i \c{c} ga:p d5 "nO5tv\i n d@N "kampf "aUf. nun E5"vE5mt@ d\i{} "zOn@ d\i{} "lUfp m\i t i5n "fKO\i ntl\i \c{c}n "StKa:ln, Un SonaX "ven\i gN "aUgN""bl\i kN tsok d5 "vand@K5 za\i m "mantl aUs. da mUste d5 "nO5tv\i n "tsugebm, das d\i{} "zOn@ f@n im "ba\i dn d5 "StE5k@K@ va@. 
%\end{IPA}

\subsection{Koreanisch}

\subsubsection{Orthografische Version}
바람과 햇님이 서로 힘이 더 세다고 다투고 있을 때, 한 나그네가 따뜻한 외투를 입고 걸어 왔습니다. 그들은 누구든지 나그네의 외투를 먼저 벗기는 이가 힘이 더 세다고 하기로 결정했습니다. 북풍은 힙껏 불었으나 불면 불수록 나그네는 외투를 단단히 여몄습니다. 그 때에 햇님이 뜨거운 햇빛을 가만히 내려쬐니, 나그네는 외투를 얼른 벗었습니다. 이리하여 북풍은 햇님이 둘중에 힘이 더 세다고 인정하지 않을 수 없었습니다.

\subsubsection{Enge Transkription der Aufnahme}
%\tolerance=10000

%\begin{IPA}
\r*ba"Ramgwa "hEn\textltailn imi \r*z2"Ro \c{c}i"mi "d2 "\r*ze:dago \r*da"t\super{h}ugo i"sW\textrtaill{} tE, "han na"gWnega ta"tWt\super{h}an "we:t\super{h}uRW\textrtaill{} "i\r*bko "\r{g}@:R2 wa"sWm\textltailn ida. \r{g}W"dWRWn nu"gudWn\textbardotlessj i na"gWneWi "we:t\super{h}uRmW\textrtaill{} "m2n\textbardotlessj 2 "\r*b2\r*dkinWn \textltailn iga \c{c}i"mi "\r*d2 "\r*ze:dago ha"giRo "\r{g}j2\textrtaill c2NhEsWm\textltailn ida. \r*bu\r{g}p\super{h}uNWn "\c{c}imk2\r*d \r*bu"R2sWna "\r*bu:\textrtaill mj2n "\r*bu:\textrtaill suro\r{g} na"gWnEnWn "we:t\super{h}uRW\textrtaill{} "\r*dadanHi j2"mj2sWm\textltailn ida. \r{g}W "tEe "hEn\textltailn imi tW"g2un hE\r*d"pic\super{h}W\textrtaill{} \r{g}a"manHi nE"Rj2 "cwe:\textltailn i na"gWnenWn "we:t\super{h}uRW\textrtaill{} "2\textrtaill \textrtaill Wn \r*b2"z2sWm\textltailn ida. i"Rihaj2 "\r*bu\r{g}phuNWn "hEn\textltailn imi "\r*du:\textrtaill cuNe \c{c}i"mi "\r*d2 "\r*ze:dago "iN\textbardotlessj 2Nha\textbardotlessj i a"nW\textrtaill{} su "@:\r*bs2sWm\textltailn ida.
\end{IPA}
\end{CJK}
\end{document}

Kompilation
===========

* cjkutf8:	$ pdflatex ipa.tex
* hangul:	$ lambda ipa.tex	$ dvips ipa.dvi	$ ps2pdf ipa.ps 


Quellen
=======

* /usr/share/texlive/2008/texmf-dist/doc/fonts/tipa/tipaman.pdf

* "Handbook of the International Phonetic Association" p123
