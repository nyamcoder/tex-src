%%% stats-sw.tex :: mito / nyamcoder %%%
\documentclass[a4paper,11pt]{article}
\usepackage{hangul}	
%\usepackage{multirow}	% Alternative für mehrzeilige Zellen 
%\usepackage{colortbl}	% für farbige Tabellenhintergründe
%\newlength\savedwidth	% nach lb2-274 "Linien variabler Stärke" 
%\newcommand\whline{\noalign{\global\savedwidth\arrayrulewidth
%					\global\arrayrulewidth 1pt}%
%				\hline
%				\noalign{\global\arrayrulewidth\savedwidth}}

%\definecolor{담청색}{cmyk}{.05,0,0,0}	% Abkürzung für \rowcolor[cmyk]{.05,0,0,0}
%\definecolor{암청색}{cmyk}{.2,0,0,0}	% Abkürzung für \rowcolor[cmyk]{.2,0,0,0}

%\usepackage{color}
%\usepackage{array}


\pagestyle{empty}

\begin{document}
\자모고딕
\centering{\large 표1. 한국 접촉 경험 및 경로}
\고딕

%\arrayrulecolor{blue}
\begin{tabular}{r|c|c|c|c}		

\multicolumn{5}{r}{\footnotesize 단위: \%}\\\cline{1-5}
%\rowcolor{암청색}
 	&						& \multicolumn{3}{c}{국가별}\\
\cline{3-5}		%\arrayrulecolor{암청색}{2-2}
%\rowcolor{암청색}
	& \raisebox{1.75ex}[0cm][0cm]{전체} 	& 미국 	& 독일	& 일본	\\\hline
%\rowcolor{담청색}
\multicolumn{1}{l|}{한국관련 뉴스접촉 정도}	& & & & \\\hline
자주 접하고 있다 	& 35	& 39	& 15	& 51	\\\hline 
접하고 있지 않다	& 64 	& 60	& 83	& 49	\\\hline 
%\rowcolor{담청색}
\multicolumn{1}{l|}{한국관련 직접경험}	& & & & \\\hline
한국 방문경험 비율	& 9	& 4	& 2	& 21	\\\hline
한국제품 구입경험자 비율 & 46 & 57 & 34	& 48	\\\hline
한국관련행사 참여경험 비율 & 11 & 12 & 10 & 11	\\\hline
알고 지내는 한국사람 있는 비율 & 31 & 42 & 20 & 30	\\\hline
\multicolumn{5}{r}{\명조\footnotesize 주: 모름/무응답 비율 제외}\\
\end{tabular}

\end{document}



* latex-graphics companion p746ff, lb2 p274 (linien variabler stärke)
* dasselbe wie stats.tex, aber die farbrelevanten elemente auskommentiert


* kompilation:

$ lambda stats-sw.tex && dvipdfmx stats.dvi


* quelle:

http://www.hankookresearch.co.kr/renewal/lesson/data/winter2005.pdf -- p9
bzw.
http://74.125.77.132/search?q=cache:HYYVrSRI9u0J:www.hankookresearch.co.kr/renewal/lesson/data/winter2005.pdf+%EB%85%B8%EC%9D%B8+1%EB%AA%85%EA%B0%80%EA%B5%AC+korean+statistics&cd=15&hl=de&ct=clnk


주: 모름/무응답 비율 제외

단위 : %
주: 모름/무응답 비율 제외
전체
국가별
미국 독일 일본
한국관련 뉴스접촉 정도
자주 접하고 있다 35 39 15 51
접하고 있지 않다 64 60 83 49
한국관련 직접경험
한국 방문경험 비율 9 4 2 21
한국제품 구입경험자 비율 46 57 34 48
한국관련행사 참여경험 비율 11 12 10 11
알고 지내는 한국사람 있는 비율 31 42 20 30


