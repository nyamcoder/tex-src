%%%%% Detailed map of Korea, main document
%%%  korea2.tex :: mito / nyamcoder
%%%  *** main config & input-files coordination ***
\documentclass{article}
\pagestyle{empty}
\usepackage[svgnames]{xcolor}
\usepackage{pst-map2dII,pst-text}
\usepackage{kotex}
\DeclareTextFontCommand{\textSEOM}{\ttfamily\SetAdhocFonts{utbm}{utgt}}	% 봄, kotexguide.pdf-39f
\DeclareTextFontCommand{\textGANG}{\color{blue}\scriptsize\SetAdhocFonts{utbt}{utbt}} % 명조 
\DeclareTextFontCommand{\textSANMAEK}{\color{Sienna}\itshape\selectfont}
\usepackage{protosem}	% \Asade의 기호 --> \byeo,\cha
\usepackage{textcomp}	% \textleaf의 기호 --> \dambae
\usepackage{dingbat}	% \anchor의 기호 조선 (또는 \newpsfontdot{Anchor}[2 0 0 2 -.78 -.7]{dingbat}{<4D>})
\usepackage{marvosym}	% \Industry의 기호; \Circpipe 기호 --> \kong
\usepackage{graphicx}	% \raisebox과 \scalebox의 명령 등

\begin{document}
%\scalebox{.8}{ %  한 페이지로 만듬 --> vs. scale! (또는 페이지 디자인의 변함)
\begin{pspicture*}(-3,-93)(11.5,-69.7)	% postscript의 그림-- 지도에서 한국 지방 부분의 좌표들

%%%%%% layer1 -- tracks, terrain / 가장 밑게 되는 층

%%%%%% korea2, layer1a -- agricultural  
%%%   kagro.tex :: mito / nyamcoder

\psset{unit=1}

%%% wheat NK // 밀-- 황해남도
\pscustom[linestyle=none]{
\psline(1.77,-81.53)(1.74,-81.35)(2,-81.3)(2.4,-81.1)(2.7,-80.6)(2,-80.3)(1.75,-80.3)(1.5,-79.8)(1.6,-79.6)(1.5,-79.4)(1.2,-79.3)(1.1,-79)(1,-78.4)(.8,-77.9)(.7,-78)(.65,-78.3)(.6,-78.9)(.55,-79)(.4,-79.5)(.8,-79.4)(.6,-79.45)(.5,-79.5)(.69,-79.52)(.73,-79.6)(1,-79.8)(1,-79.92)(.7,-79.75)(.3,-79.8)(.25,-80)(0,-80.25)(-0.05,-80.2)(-.19,-80.4)(-.2,-80.6)(-.1,-80.6)(.3,-80.8)(.45,-80.9)(.4,-81.05)(.3,-81)(.2,-81)(.2,-81.1)(.32,-81.12)(.25,-81.18)(.1,-81.15)(0,-81.2)(0,-81.25)(.34,-81.3)(.37,-81.2)(.5,-81.3)(.7,-81.22)(.85,-81)(1.1,-81)(1.2,-81.05)(1.49,-81.23)(1.52,-81.23)(1.54,-81.3)(1.62,-81.28)(1.66,-81.4)(1.68,-81.43)(1.7,-81.52)(1.77,-81.53)
\fill[fillstyle=solid,fillcolor=LemonChiffon]
\fill[fillstyle=vlines,hatchcolor=Magenta,hatchwidth=.5pt]}
%%% `wheatfree' areas // `밀의 구멍´ (평양)
\psccurve*[linecolor=white](1.75,-80.9)(1.9,-80.8)(1.2,-80.5)(1.5,-80.8)
\psccurve*[linecolor=white](.6,-80)(.1,-80.6)(.5,-80.3)
\psccurve*[linecolor=white](.7,-80.6)(.3,-80.8)(.6,-80.9)(1,-80.9)

%%% 밀--경상복도1
\pscustom[linestyle=none]{
\psccurve(5.2,-83.9)(5.4,-83.8)(5.45,-83.7)(5.5,-83.5)(5.7,-83.5)(5.8,-83.3)(5.7,-83.3)(5.3,-83.45)(5.2,-83.75)
\fill[fillstyle=solid,fillcolor=LemonChiffon]
\fill[fillstyle=vlines,hatchcolor=Magenta,hatchwidth=.5pt]}
%%% 밀--경상복도2
\pscustom[linestyle=none]{
\psccurve(5.7,-83.7)(6.1,-83.6)(6,-83.5)(6.1,-83.4)(6.1,-83.3)(5.8,-83.5)
\fill[fillstyle=solid,fillcolor=LemonChiffon]
\fill[fillstyle=vlines,hatchcolor=Magenta,hatchwidth=.5pt]}

%%%  sorghum/수수--경상북도1
\pscustom[linestyle=none]{
\psccurve(5.2,-83.75)(5.2,-83.45)(5.1,-83.55)(5,-83.9)
\fill[fillstyle=solid,fillcolor=Bisque]
\fill[fillstyle=hlines,hatchcolor=PaleGreen,hatchwidth=.5pt]}
%%% 수수--경상북도2
\pscustom[linestyle=none]{
\psccurve(6.4,-83.4)(6.5,-83.2)(6.45,-83.7)(6.3,-83.7)(6.05,-83.65)(6.3,-83.5)
\fill[fillstyle=solid,fillcolor=Bisque]
\fill[fillstyle=hlines,hatchcolor=PaleGreen,hatchwidth=.5pt]}
%%% 수수--경상북도3
\pscustom[linestyle=none]{
\psccurve(6,-83.7)(6.1,-84.1)(6.35,-84.1)(6.35,-84.25)(6.6,-84.25)(6.6,-84.4)(6.3,-84.4)(6.1,-84.4)(6.1,-84.2)(5.9,-84.2)(5.9,-83.7)
\fill[fillstyle=solid,fillcolor=Bisque]
\fill[fillstyle=hlines,hatchcolor=PaleGreen,hatchwidth=.5pt]}

%%% 수수--황해복도
\pscustom[linestyle=none]{
\pscurve(2.4,-81.1)(2.2,-81.25)(2.35,-81.4)(2.5,-81.45)(2.8,-81.3)(3,-81.5)(3.2,-81.2)(3.5,-80.9)(3.9,-80.3)(3.4,-79.8)(3.2,-80.6)(3,-80.5)(2.85,-80.3)(3,-79.8)(3,-79.5)(2.95,-79.7)(2.9,-79.8)(2.7,-80.2)(2.6,-79.5)(2.8,-79)(2.5,-79.2)(2,-79)(1.6,-78.8)(1.6,-78.7)(2,-78.5)(2.2,-78)(2.15,-77.7)(2.4,-77.4)(2,-77.5)(1.4,-77.6)(.97,-77.8)(1.05,-78.5)(1.2,-79.2)(1.6,-79.1)(1.7,-79.1)(2.2,-79.25)(2.5,-79.4)(2.3,-79.5)(2,-79.3)(1.7,-79.3)(2.3,-79.7)(2.4,-80)(2.5,-80.25)(2.65,-80.5)(2.4,-81.1)
\fill[fillstyle=solid,fillcolor=Bisque]
\fill[fillstyle=hlines,hatchcolor=PaleGreen,hatchwidth=.5pt]}

%%% 수수--평양
\pscustom[linestyle=none]{
\psccurve(2,-80)(1.7,-79.5)(1.4,-79.3)(1.5,-79.8)
\fill[fillstyle=solid,fillcolor=Bisque]
\fill[fillstyle=hlines,hatchcolor=PaleGreen,hatchwidth=.5pt]}

%%% 수수--원산
\pscustom[linestyle=none]{
\psccurve(3.7,-78.8)(3.5,-78.5)(3.55,-78.35)(3.7,-78.2)(3.75,-77.8)(3.6,-77.6)(3.5,-77.5)(3.3,-78.3)(3.4,-78.5)(3.5,-78.6)
\fill[fillstyle=solid,fillcolor=Bisque]
\fill[fillstyle=hlines,hatchcolor=PaleGreen,hatchwidth=.5pt]}


%%%%%% korea2,  layer1b -- railroad system
%%%   ktracks.tex :: mito / nyamcoder

%\psset{unit=1} % for standalone

%%% '철도 호선' / virtual railroad lines
% (1) 부산--하어빈    Busan--Harbin
\pscustom[linecolor=white,linewidth=.5pt,linestyle=dashed,dash=4pt 4pt,linearc=.2]{%
\psline
(6.45,-86.5)(6.4,-86)(6.2,-85.9)(5.9,-85.8)(6,-85.3)(5.7,-85.03)% 부산--대구
(5.2,-84.6)(5,-84.65)(4.8,-84.6)(4.7,-84.4)(4.3,-84.5)(4,-84.42)(3.9,-84.45)(3.8,-84.35)(3.75,-84.2)% 대전 
(3.5,-83.6)(3.25,-83.5)(3.12,-82.45)% 수원
(3.1,-82)(3.05,-81.9)% 서울
(3.2,-81.5)(3.35,-80.5)(3.4,-80)(3.75,-78.9)(3.7,-78.75)% 원산
(3.53,-78.5)% 문천
(3.55,-78.3)(3.57,-78.2)(3.7,-77.6)(3.85,-77.35)% 흥남
(4.25,-77.2)(4.5,-76.9)(4.85,-76.85)% 신포 
(5.35,-76.3)(5.8,-76.05)% 단천 
(6.2,-75.63)% 김책
(6.7,-74.3)(6.6,-73.8)(6.95,-73.35)% 청진
(7.2,-72.5)(6.95,-71.1)(6.9,-71)% 연길 
(6.43,-70.96)(6.3,-70.72)(6.2,-70.5)(6.1,-69.6)% --> 하어빈,블라디보스토크 
\stroke[linewidth=3\pslinewidth,linecolor=black,linestyle=solid]}


% (2) 목포--대전    Mokpo--Daejeon
\pscustom[linecolor=white,linewidth=.5pt,linestyle=dashed,dash=4pt 4pt,linearc=.2]{%
\psline(2.2,-87.3)(2.5,-87.1)(2.65,-86.85)(2.7,-86.8)(2.75,-86.6)% 목포--광주
(2.5,-86.4)(3.2,-85.25)% 전주
(3.3,-84.6)(3.45,-84.35)(3.6,-84.3)(3.65,-84.3)(3.75,-84.2)% --대전  
\stroke[linewidth=3\pslinewidth,linecolor=black,linestyle=solid]}


% (3) 서울--신의주    Seoul--Sinŭiju
\pscustom[linecolor=white,linewidth=.5pt,linestyle=dashed,dash=4pt 4pt,linearc=.15]{%
\psline(3.05,-81.9)(2.9,-81.2)(2.3,-81.15)% 서울--개성
(2.5,-80.65)(2.3,-80.2)(1.5,-80.1)(1.15,-79.25)(1.1,-79.1)% --평양
(1.05,-78.5)(.9,-77.95)% 안주 
(.7,-77.85)(.6,-77.9)(.5,-77.95)(.3,-77.9)(0,-77.8)(-.3,-77.6)(-.5,-77.5)(-.6,-77.4)
(-.65,-77.3)(-.7,-77.1)(-.9,-76.95)% 신의주
\stroke[linewidth=3\pslinewidth,linecolor=black,linestyle=solid]}


% (4) 신의주--톈진    Sinŭiju--Tianjin
\pscustom[linecolor=white,linewidth=.5pt,linestyle=dashed,dash=4pt 4pt,linearc=.25]{%
\psline(-.9,-76.95)(-1.2,-76.5)(-1.4,-76)(-1.6,-75.7)(-1.65,-75)(-1.7,-73.9)(-1.8,-73.8)(-1.8,-73.4)% 신의주--쑤자툰--선양  
(-2.5,-72.8)(-3,-72.9)% -->톈진
\stroke[linewidth=3\pslinewidth,linecolor=black,linestyle=solid]}


% (5) 안주--랴오위안--쓰핑    Anju--Liaoyuan--Siping
\pscustom[linecolor=white,linewidth=.5pt,linestyle=dashed,dash=4pt 4pt,linearc=.1]{%
\psline(.9,-77.97)(.95,-77.8)(1.35,-77.75)(1.7,-77.3)(1.88,-76.6)% 희천
(2.1,-76)(2.5,-75.4)(2.33,-75.2)% 강계
(2.1,-75.1)(2,-74.95)(1.9,-74.87)% 만포
(1.6,-75)(1.9,-74.3)(1.6,-73.6)(1.4,-73.7)% 통화
(1.1,-73.6)(1,-73.5)(1.1,-73.3)(1.15,-73.1)(1.1,-72.9)(1.05,-72.8)(1,-72.7)(1.1,-72.2)(.2,-71.5)% 랴오위안
(0,-71.47)(-.2,-71.42)(-.4,-71.3)(-.9,-71)% 쓰핑
(-1.2,-70.7)(-1.5,-70.5)(-2.2,-70)(-3,-69.5)% -->... 
\stroke[linewidth=3\pslinewidth,linecolor=black,linestyle=solid]}


% (6) 쑤자툰--다롄     Sujiatun--Dalian
\pscustom[linecolor=white,linewidth=.5pt,linestyle=dashed,dash=4pt 4pt,linearc=.4]{%
\psline(-1.8,-73.77)(-2,-74.3)(-2.2,-74.6)(-2.4,-74.75)(-2.7,-74.9)(-3.05,-74.95)% 쑤자툰--안산 -->다롄
\stroke[linewidth=3\pslinewidth,linecolor=black,linestyle=solid]}

	
% (7) 선양--(랴오위안)--창춘    Shenyang--(Liaoyuan)--Changchun
\pscustom[linecolor=white,linewidth=.5pt,linestyle=dashed,dash=4pt 4pt,linearc=.05]{%
\psline(-1.8,-73.6)(-1.6,-73.5)% 푸순 (Fushun)
(-1.5,-73.45)(-1.2,-73.4)(-1,-73.35)(-.8,-73.25)
(-.5,-73.2)(-.3,-73.1)(-.1,-73.05)(.15,-72.9)
(.7,-72.5)(.95,-72.3)(.95,-72.2)(1.1,-72.1)
(1.4,-71.95)(1.53,-71.95)
(1.8,-71.7)(1.8,-71.6)(1.9,-71.4)
(2.1,-71.3)(2,-71.25)(1.9,-71.1)
(1.7,-71)(1.55,-70.8)(1.5,-70.72)(1.48,-70.6)(1.25,-70.3)(1.3,-70.1)(1.12,-69.5)
\stroke[linewidth=3\pslinewidth,linecolor=black,linestyle=solid]}


% (8) 선양--쓰핑--창춘    Shenyang--Siping--Changchun
\pscustom[linecolor=white,linewidth=.5pt,linestyle=dashed,dash=4pt 4pt,linearc=.05]{%
\psline(-1.8,-73.4)(-1.35,-72.2)(-.93,-71)% 쓰핑
(-.4,-70.4)(.2,-69.5)
\stroke[linewidth=3\pslinewidth,linecolor=black,linestyle=solid]}


% (9) 블라디보스토크    Vladivostok
\pscustom[linecolor=white,linewidth=.5pt,linestyle=dashed,dash=4pt 4pt,linearc=.2]{%
\psline(9.75,-70.4)(10,-70)(9.75,-69.65)
\stroke[linewidth=3\pslinewidth,linecolor=black,linestyle=solid]}





%%%%%% layer2 -- map essentials / 지도의 층

%%% 기본적인 지도의 그리기
\psset{path=./dataII}
\psset{unit=39,type=8,latitude0=133,longitude0=125}
\WorldMapII[asia,samer=false,namer=false,europe=false,africa=false,maillage,increment=5,linecolor=DarkBlue,linewidth=0.75\pslinewidth,borders,rivers]

%%%  잘못 만든 (남한·낙동) 강선의 삭제
\psset{unit=1cm}
\psccurve*[linecolor=white](6.2,-82.5)(6,-82.1)(6.3,-82.3)


%%%%%% layer3 -- div. data / 자료 

%%%%%% korea2, layer2b -- mountain ranges, capitals
%%%   kextra.tex :: mito / nyamcoder

%\psset{unit=1cm}  % for standalone

%%% 산맥   
%\DeclareTextFontCommand{\textSANMAEK}{\color{Sienna}\itshape\selectfont}  ----> main doc preamble only!

% Changbai Mountains    \large [linecolor=Brown]  
\pstextpath{\pscurve[linestyle=none](2.1,-73.7)(2.7,-73.2)(3.45,-72.85)(4.5,-72.55)(5,-72.5)}{\color{SlateGray}\it\shortstack{\qquad 장백산맥\\\small{(창바이 산맥)\,/}}}  
\pstextpath{\pscurve[linestyle=none](3,-73.3)(3.8,-73.65)(4.5,-73.7)}{\color{SlateGray}\it 백두산맥}  

% Gaema Plateau
\rput(4.2,-75.8){\textSANMAEK{\small 개마고원}}

% Gangnam Mountains 
\rput{30}(.75,-76.2){\textSANMAEK{\small 강\,남\,산\,맥}}

% Hamgyeong Mountains 
\rput{55}(5.1,-75.75){\textSANMAEK{\small 함경산맥}}

% Jeogyuryeong Mountains 
\rput{25}(2.2,-76.1){\textSANMAEK{\small 저\,규\,령\,산\,맥}}

% Jiri Mountains 
\rput{70}(4.1,-86){\textSANMAEK{지}}
\rput{55}(4.3,-85.6){\textSANMAEK{리}}
\rput{42}(4.57,-85.23){\textSANMAEK{산}}
\rput{25}(4.95,-84.95){\textSANMAEK{맥}}

% Macheollyeong Mountains 
\rput{-70}(5.5,-74.5){\textSANMAEK{\small 마철령산맥}}

% Myohyang Mountains 
\rput{45}(2.85,-77){\textSANMAEK{\small 묘\,향\,산\,맥}}

% Nangnim Mountains (aka 랑림산맥)
\rput{-80}(3.3,-75.6){\textSANMAEK{\small 낭\,림\,산\,맥}} % 참고 랑림산맥 

% Sobaek Mountains 
\rput{60}(4.6,-84.1){\textSANMAEK{소}}
\rput{45}(4.9,-83.7){\textSANMAEK{백}}
\rput{30}(5.3,-83.3){\textSANMAEK{산}}
\rput{22}(5.75,-83.05){\textSANMAEK{맥}}

% Taebaek Mountains
\rput{-45}(4.5,-79.7){\textSANMAEK{\large 태}}
\rput{-65}(6.2,-82){\textSANMAEK{\large 백}}
\rput{-90}(6.5,-83.7){\textSANMAEK{\large 산}}
\rput{-110}(6.3,-85.5){\textSANMAEK{\large 맥}}


%%% capitals/특별한 중심 도시의 방식
\psset{unit=39}	% 또는 unit=39cm; zoom 39x 
\pnodeMap(125.73,39.03){Pyeongyang}
  \uput[0](Pyeongyang){\colorbox{OrangeRed}{\sf\textbf{평양}}}

%  \psdot[linecolor=red,dotscale=1.5](Pyeongyang)  
  \psdot[dotstyle=Bo,dotscale=1.5,fillcolor=red](Pyeongyang)  
  \psdot[dotscale=.3](Pyeongyang)  
\pnodeMap(126.983333,37.55){Seoul}
  \uput[80](Seoul){\colorbox{OrangeRed}{\sf\textbf{서울}}}
  
%  \psdot[linecolor=red,dotscale=1.5](Seoul)  
  \psdot[dotstyle=Bo,dotscale=1.5,fillcolor=red](Seoul)  
  \psdot[dotscale=.3](Seoul)  


%%%%%% korea2, layer2c -- discrete places
%%%   kplaces.tex :: mito / nyamcoder

%%% 장소 놓기 모양의 치수; Herbert Voss 의 http://tug.org/PSTricks/pst-geo/DE.tex파일으로  찾아서
% \psset{unit=39}

%%% cities NK / 북한의 도시
% formatting:
\makeatletter
\def\bukhandosi{\@ifnextchar*{\@startrue\bukhandosi@i}{\@starfalse\bukhandosi@i*}}
\def\bukhandosi@i*{\@ifnextchar[{\bukhandosi@ii}{\bukhandosi@ii[DarkRed]}}
\def\bukhandosi@ii[#1]{\@ifnextchar<{\bukhandosi@iii[#1]}{\bukhandosi@iii[#1]<0>}}%
\def\bukhandosi@iii[#1]<#2>(#3,#4,#5,#6){%
	\def\pst@tempA{#6}
	\ifx\pst@tempA\empty\def\pst@tempB{#5}\else\def\pst@tempB{#6}\fi
	\pnodeMap(#3,#4){\pst@tempB}
	\if@star\uput*[#2](\pst@tempB){\color{#1}\sffamily\bfseries#5}
	\else\uput[#2](\pst@tempB){\color{#1}\sffamily\bfseries#5}\fi
	\psdot[linecolor=DarkRed](\pst@tempB)\ignorespaces}
\makeatother

% \bukhandosi(125.753765,39.031859,평양,Pyeongyang) --> kextra
\bukhandosi<145>(125.66,39.62,안주,Anju)
\bukhandosi(129.4,42,부윤,Buyun)
\bukhandosi(129.783333,41.8,청진,Cheongjin)
\bukhandosi(128.911,40.458,단천,Dancheon)
\bukhandosi(126.312,39.761,덕천,Deokcheon)
\bukhandosi<105>(125.906111,39.698611,개천,Gaecheon)
\bukhandosi<90>(126.55,37.966667,개성,Gaeseong)
\bukhandosi<-145>(126.6,40.966667,강계,Ganggye)
\bukhandosi(129.200556,40.667222,김책,Gimchaek)
\bukhandosi<-90>(125.716667,38.033333,해주,Haeju)
\bukhandosi<120>(127.536667,39.913056,함흥,Hamheung)
\bukhandosi<-30>(127.6,39.85,흥남,Heungnam)
\bukhandosi<-90>(129.75,42.433333,회령,Hoeryong)
\bukhandosi<180>(126.25,40.25,희천,Huicheon)
\bukhandosi<-65>(128.183333,41.4,혜산,Hyesan)
\bukhandosi(126.29,41.157,만포,Manpo)
\bukhandosi<155>(127.356,39.259,문천,Muncheon)
\bukhandosi(125.4,38.733333,남포,Nampo)
\bukhandosi(125.85,39.25,평성,Pyeongseong)
\bukhandosi<-45>(130.384444,42.344444,나선(라선),Raseon) % aka 라진- 선봉시 (~2001년)
\bukhandosi<20>(125.759722,38.506389,사리원,Sariwon)
\bukhandosi(124.4,40.1,신의주,Sineuiju)
\bukhandosi(128.25,40.083333,신포,Sinpo)
\bukhandosi*(127.446111,39.1475,원산,Wonsan) %  별을 사용하기는 도시 레이블의 바탕색을 덮기 / solid white bg demo

%%% cities SK / 남한의 도시
% formatting:
\makeatletter
\def\namhandosi{\@ifnextchar*{\@startrue\namhandosi@i}{\@starfalse\namhandosi@i*}}
\def\namhandosi@i*{\@ifnextchar[{\namhandosi@ii}{\namhandosi@ii[DarkRed]}}
\def\namhandosi@ii[#1]{\@ifnextchar<{\namhandosi@iii[#1]}{\namhandosi@iii[#1]<0>}}%
\def\namhandosi@iii[#1]<#2>(#3,#4,#5,#6){%
	\def\pst@tempA{#6}
	\ifx\pst@tempA\empty\def\pst@tempB{#5}\else\def\pst@tempB{#6}\fi
	\pnodeMap(#3,#4){\pst@tempB}
	\if@star\uput*[#2](\pst@tempB){\color{#1}\sffamily\bfseries#5}
	\else\uput[#2](\pst@tempB){\color{#1}\sffamily\bfseries#5}\fi
	\psdot[linecolor=DarkRed](\pst@tempB)\ignorespaces}
\makeatother

% \namhandosi(126.983333,37.55,서울,Seoul) --> kextra
\namhandosi<48>(128.716667,36.566667,안동,Andong)
\namhandosi<-45>(129.033333,35.1,부산,Busan)
\namhandosi(128.663056,35.270833,창원,Changweon)
\namhandosi<10>(127.483333,36.633333,청주,Cheongju)
\namhandosi(127.733333,37.866667,춘천,Chuncheon)
\namhandosi<-10>(128.6,35.866667,대구,Daegu)
\namhandosi<-90>(127.3849,36.3501,대전,Daejeon)
\namhandosi(128.9,37.75,강릉,Gangneung)
\namhandosi<110>(127.116667,36.45,공주,Gongju)
\namhandosi(126.716667,35.983333,군산,Gunsan)
\namhandosi(126.916667,35.166667,광주,Gwangju)
\namhandosi<135>(126.633333,37.483333,인천,Incheon)
\namhandosi<160>(126.517,33.48,제주,Jeju City)
\namhandosi(127.15,35.816667,전주,Jeonju)
\namhandosi(126.35,34.766667,목포,Mokpo)
\namhandosi(129.365,36.032222,포항,Pohang)
\namhandosi(129.165,37.45,삼척,Samcheok)
\namhandosi<-45>(126.6,33.3,서귀포,Seogwipo) % 126.666667,33.366667
\namhandosi(128.566667,38.2,속초,Sokcho)
\namhandosi<-45>(127.016667,37.266667,수원,Suwon)
\namhandosi(128.983333,37.166667,태백,Taebaek)
\namhandosi(129.316667,35.55,울산,Ulsan)
\namhandosi(127.783333,37.216667,원주,Wonju)
\namhandosi<-45>(127.665278,34.762778,여수,Yeosu)


%%% international cities / 국제 도시
% formatting:
\makeatletter
\def\gukchedosi{\@ifnextchar*{\@startrue\gukchedosi@i}{\@starfalse\gukchedosi@i*}}
\def\gukchedosi@i*{\@ifnextchar[{\gukchedosi@ii}{\gukchedosi@ii[black]}}
\def\gukchedosi@ii[#1]{\@ifnextchar<{\gukchedosi@iii[#1]}{\gukchedosi@iii[#1]<0>}}%
\def\gukchedosi@iii[#1]<#2>(#3,#4,#5,#6){%
	\def\pst@tempA{#6}
	\ifx\pst@tempA\empty\def\pst@tempB{#5}\else\def\pst@tempB{#6}\fi
	\pnodeMap(#3,#4){\pst@tempB}
	\if@star\uput*[#2](\pst@tempB){\color{#1}\sffamily\bfseries#5}
	\else\uput[#2](\pst@tempB){\color{#1}\sffamily\bfseries#5}\fi
	\psdot(\pst@tempB)\ignorespaces}
\makeatother

%%% cities Japan / 외국도시-- 일본
\gukchedosi<-65>(130.4,33.583333,후쿠오카,Fukuoka)
\gukchedosi(132.45,34.383333,히로시마,Hiroshima)

%\gukchedosi(130.55,31.6,가고시마,Kagoshima)
\gukchedosi<-60>(130.883333,33.883333,기타큐슈,Kitakyushu)
\gukchedosi(130.733333,32.783333,구마모토,Kumamoto)
\gukchedosi<90>(129.866667,32.783333,나가사키,Nagasaki)
\gukchedosi<35>(130.933333,33.95,시모노세키,Shimonoseki)
\gukchedosi<90>(131.466667,34.183333,야마구치,Yamaguchi)

%%% cities China / 외국도시-- 중국
\gukchedosi<-45>(122.99,41.108333,안산,Anshan)
\gukchedosi[DarkSeaGreen]<-80>(128.316667,42.55,안투(안도),Antu)
%\gukchedosi(125.316667,43.883333,창춘,Changchun)
\gukchedosi<180>(124.383333,40.116667,단둥,Dandong)
\gukchedosi[DarkSeaGreen]<90>(128.225,43.358333,둔화(돈화),Dunhua)
\gukchedosi<65>(123.9,41.866667,푸순,Fushun)
\gukchedosi(126.633333,45.75,하얼빈,Harbin)
\gukchedosi[DarkSeaGreen]<70>(129.008499,42.54221,허룽,Helong)
%\gukchedosi(126.566667,43.866667,지린,Jilin)
\gukchedosi<-90>(125.140278,42.900556,랴오위안,Liaoyuan)
\gukchedosi(129.599722,44.586111,무단장,Mudanjiang)
\gukchedosi<180>(123.75,41.91,선양,Shenyang)
%\gukchedosi(124.368611,43.163333,쓰핑,Siping)
\gukchedosi<-90>(125.933333,41.716667,퉁화,Tonghua)
\gukchedosi[DarkSeaGreen]<40>(129.843,42.966,투먼(도문),Tumen)
\gukchedosi[DarkSeaGreen]<-160>(129.5,43.05,옌지(연길),Yanji)

%%% cities Russia / 외국도시-- 러시아
\gukchedosi<-120>(132.883333,42.816667,나홋카,Nakhodka)
\gukchedosi<75>(131.9,43.133333,블라디보스토크,Vladivostok)


%%% islands & islets / 섬
% is text only
\pnodeMap(125.77,33.3){Jejudo}\rput(Jejudo){\color{DarkRed}\textSEOM{제주도}} %(126.528056,33.365556)
%\pnodeMap(127,33.5){Udo}\rput(Udo){\color{DarkBlue}\textbf{.}}
\pnodeMap(127.11,33.57){U-do}\rput(U-do){\color{DarkRed}\textSEOM{\scriptsize 우도}}
\pnodeMap(130.4,37.483333){Ulleungdo} % (130.9,37.483333)
\pnodeMap(131.867778,37.240833){Dokdo}\rput(Dokdo){\color{DarkBlue}\textbf{.}} % Liancourt Rocks
\pnodeMap(131.86,37.15){Dok-do}\rput(Dok-do){\color{DarkRed}\textSEOM{\scriptsize 독도}}
\pnodeMap(125.33,34){Gageo-do}
	\rput(Gageo-do){\color{DarkRed}\textSEOM{\shortstack{\tiny 가거도\\[-.2em]\tiny (소흑산도)}}}
\pnodeMap(126.27,33.15){Gapado}\rput(Gapado){\color{DarkBlue}\textbf{.}}
\pnodeMap(126.07,33.12){Gapa-do}\rput(Gapa-do){\color{DarkRed}\textSEOM{\tiny 가파도}}
\pnodeMap(126.28,33.07){Marado}\rput(Marado){\color{DarkBlue}.}
\pnodeMap(126.48,33.05){Mara-do}\rput(Mara-do){\color{DarkRed}\textSEOM{\tiny 마라도}}
\pnodeMap(125.18,32.1){Ieodo}\rput(Ieodo){.} % Socotra Rock
\pnodeMap(125.28,32.18){Ieo-do}\rput(Ieo-do){\color{DarkRed}\textSEOM{\tiny 이어도}}
\pnodeMap(129.55,34.35){Tsushima}\rput{74}(Tsushima){\textSEOM{쓰시마(대마도)}} % (129.326944,34.416667)


%%% nations/ 나라
% is text only
%
%\pnodeMap(128.5,37){South Korea}
%\rput(South Korea){\huge\textsf{\textcolor{LightPink}{한국}}}
%\pnodeMap(127.5,40.5){North Korea}
%\rput(North Korea){\huge\textsf{\textcolor{LightPink}{북조선}}}
\pnodeMap(131,33.05){Japan}\rput(Japan){\huge\textsf{\textcolor{LightPink}{일본}}}
\pnodeMap(124.65,42.4){China}\rput(China){\huge\textsf{\textcolor{LightPink}{중국}}}
\pnodeMap(131.25,42.75){Russia}\rput{40}(Russia){\huge\textsf{\textcolor{LightPink}{러시아}}}


%%% sea/ 바다
% is text only
\pnodeMap(131,39){Japanese Sea}\rput(Japanese Sea){\Huge\it\textcolor{SkyBlue}{동해}}
\pnodeMap(123.8,37.8){Yellow Sea}\rput(Yellow Sea){\LARGE\it\textcolor{SkyBlue}{\shortstack{서해\\(황해)}}}
\pnodeMap(128,33.3){East China Sea}\rput(East China Sea){\huge\it\textcolor{SkyBlue}{남해}}
\pnodeMap(129.4,34.5){Korea Strait} % (129.796667,34.599444)
	\rput{30}(Korea Strait){\Large\it\textcolor{SkyBlue}{대한\qquad\qquad 해협}}
\pnodeMap(126.6,33.77){Jeju Strait}\rput(Jeju Strait){\large\it\textcolor{SkyBlue}{제주해협}}
\pnodeMap(124.2,39){Korea Bay}\rput(Korea Bay){\it\textcolor{SkyBlue}{\shortstack{서한만/\\서조선만}}}
\pnodeMap(128.8,39.5){East Korea Bay}\rput(East Korea Bay){\it\textcolor{SkyBlue}{\shortstack{동한만/\\동조선만}}}
\pnodeMap(125.5,37.3){Gyeonggi Bay}\rput(Gyeonggi Bay){\it\textcolor{SkyBlue}{경기만}}


%%% mountain peaks/ 산정
% formatting:
\makeatletter
\def\hansan{\@ifnextchar*{\@startrue\hansan@i}{\@starfalse\hansan@i*}}
\def\hansan@i*{\@ifnextchar[{\hansan@ii}{\hansan@ii[black]}}
\def\hansan@ii[#1]{\@ifnextchar<{\hansan@iii[#1]}{\hansan@iii[#1]<0>}}%
\def\hansan@iii[#1]<#2>(#3,#4,#5,#6,#7){%
	\def\pst@tempA{#6}
	\ifx\pst@tempA\empty\def\pst@tempB{#5}\else\def\pst@tempB{#6}\fi
	\pnodeMap(#3,#4){\pst@tempB}
	\if@star\uput*{3pt}[#2](\pst@tempB){\color{#1}\tiny\shortstack{#5\\[-.3em]#7}}
	\else\uput*{3pt}[#2](\pst@tempB){\color{#1}\tiny\shortstack{#5\\[-.3em]#7}}\fi
	\psdot*[linecolor=#1,dotstyle=triangle](\pst@tempB)\ignorespaces}
\makeatother

\hansan<90>(128.2,38,안산,Ansan,1430)
\hansan[red]<180>(128.055278,42.005556,백두산,Baekdusan,2750)
\hansan(128.05,40.78,백산,Baeksan,2379)
\hansan<45>(128.8,40.75,복개산,Bokgaesan,1565)
\hansan<90>(127.2,40.36,북산,Buksan,2070)
\hansan<90>(127.9,40.84,차일봉,Chailbong,2506)
\hansan(127.45,40.3,천불산,Cheonbulsan,1455)
\hansan(129.65,41.05,칠보산,Chilbosan,906)
\hansan<-145>(125.3,40.38,단풍덕산,Danpungdeoksan,1159)
\hansan<90>(127.8,35.9,덕유산,Deogyusan,1614)
\hansan(126.7,40.33,동백산,Dongbaeksan,2096)
\hansan[red]<-30>(128.9,41.25,두류산,Duryusan,2309)
\hansan<90>(127.8,41.15,두운봉,Duunbong,2487)
\hansan(128.051667,38.526111,금강산,Geumgangsan,1638)
\hansan<150>(129.2,41.7,관모봉,Gwanmobong,2541)
\hansan<180>(129.07,41.58,궤상봉,Gwesangbong,2333)
\hansan<90>(126.8,41.55,학송산,Haksongsan,1276)
\hansan[red]<10>(126.533333,33.366667,한라산,Hallasan,1950)
\hansan<90>(126.6,39.1,하랑산,Harangsan,1486)
\hansan<90>(127.35,41.2,희색봉,Heuisaekbong,2185)
\hansan(127.716667,35.333333,지리산,Jirisan,1915)
\hansan<10>(125.6,40.45,주사산,Jusasan,1750)
\hansan<90>(127,40.8,맹부산,Maengbusan,2214)
\hansan[red]<165>(129.1,41.35,만탑산,Mantabsan,2205)
\hansan<10>(126.333056,40.018611,묘향산,Myohyangsan,1909)
\hansan[red]<90>(128.47,41.6,남포태산,Nampotaesan,2435)
\hansan<60>(126.5,40.37,낭림산,Nangnimsan,2184) % aka 랑림산
\hansan<90>(126.65,39.4,오봉산,Obongsan,1289)
\hansan<-150>(128.7,37.6,오대산,Odaesan,1563)
\hansan<90>(127,41.3,오가산,Ogabong,1598)
\hansan(128,40.3,팔봉산,Palbongsan,1681)
\hansan(125.5,40.08,삼각산,Samgaksan,937)
\hansan[red](130.87,37.5,성인봉,Seonginbong,984)
\hansan<90>(128.483333,37.75,설악산,Seoraksan,1708)
\hansan<90>(128.47,36.97,소백산,Sobaeksan,1440)
\hansan<90>(130.25,42.45,송진산,Songjinsan,1146)
\hansan<90>(126.15,40.55,송적산,Songjeoksan,1970)
\hansan(128.75,37.17,태백산,Taebaeksan,1567)
\hansan<180>(128.25,37.4,태기산,Taegisan,1563)
\hansan<-25>(129.13,41.64,투구봉,Tugubong,2335)
\hansan<90>(128.1,36.9,월악산,Woraksan,1094)
\hansan<90>(127.5,40.9,연화산,Yeonhwasan,2355)


%%% rivers Korea/강--한국
% is text only
\pnodeMap(127.47,37.63){Bukhan River-1}\rput{-5}(Bukhan River-1){\textGANG{북}}
\pnodeMap(127.63,37.7){Bukhan River-2}\rput{55}(Bukhan River-2){\textGANG{한}}
\pnodeMap(127.75,37.78){Bukhan River-3}\rput{25}(Bukhan River-3){\textGANG{강}}
\pnodeMap(125.85,39.59){Daedong River-1}\rput{55}(Daedong River-1){\textGANG{대}}
\pnodeMap(126.08,39.65){Daedong River-2}\rput{7}(Daedong River-2){\textGANG{동강}}
\pnodeMap(127.2,36.52){Geumho River-1}\rput{35}(Geumho River-1){\textGANG{금}}
\pnodeMap(127.43,36.55){Geumho River-2}\rput{-40}(Geumho River-2){\textGANG{호}}
\pnodeMap(127.55,36.51){Geumho River-3}\rput{-10}(Geumho River-3){\textGANG{강}}
\pnodeMap(126.78,37.52){Han River}\rput{-35}(Han River){\textGANG{한강}}
\pnodeMap(127.08,38.7){Imjin River}\rput{68}(Imjin River){\textGANG{임진강}}
\pnodeMap(128.63,36.47){Nakdong River}\rput{13}(Nakdong River){\textGANG{낙동강}}
\pnodeMap(127.5,37.3){Namhan River}\rput{-50}(Namhan River){\textGANG{남한강}}
\pnodeMap(128.83,41.95){Tumen River-1}\rput{-5}(Tumen River-1){\textGANG{두}}
\pnodeMap(129.08,42.02){Tumen River-2}\rput{45}(Tumen River-2){\textGANG{만강}}
\pnodeMap(124.6,40.37){Yalu River-1}\rput{38}(Yalu River-1){\textGANG{압록강}} % 야루장
\pnodeMap(127.57,41.535){Yalu River-2}\rput{-10}(Yalu River-2){\textGANG{압}}
\pnodeMap(127.73,41.49){Yalu River-3}\rput(Yalu River-3){\textGANG{록}}
\pnodeMap(127.89,41.515){Yalu River-4}\rput{15}(Yalu River-4){\textGANG{강}}


%%% rivers China/강--중국
% is text only
\pnodeMap(123.57,42.32){Liao River-1}\rput{25}(Liao River-1){\textGANG{요하}}
\pnodeMap(123.61,42.15){Liao River-2}\rput{25}(Liao River-2){\textGANG{(랴오허강)}}
\pnodeMap(127.46,42.72){Sungari River-1}\rput{42}(Sungari River-1){\textGANG{송화}}
\pnodeMap(127.63,42.86){Sungari River-2}\rput{60}(Sungari River-2){\textGANG{강}}
\pnodeMap(127.89,42.85){Sungari River-3}\rput{-50}(Sungari River-3){\textGANG{(쑹}}
\pnodeMap(128.01,42.72){Sungari River-4}\rput{-30}(Sungari River-4){\textGANG{화}}
\pnodeMap(128.21,42.66){Sungari River-5}\rput{-10}(Sungari River-5){\textGANG{강)}}
\pnodeMap(129.64,42.7){Tumen River-3}\rput{83}(Tumen River-3){\textGANG{(투먼장)}}
\pnodeMap(125.35,40.775){Yalu River-5}\rput{32}(Yalu River-5){\textGANG{(야루장)}}

%%% EOF
%%%%%% korea2, layer2d -- diverse
%%% kdata.tex :: mito / nyamcoder

% \psset{unit=39} % for standalone

%%% rice / 벼
\newcommand\byeo{\large\textcolor{DarkOliveGreen}{\textproto{\Asade}}} % --> protosem.sty
\pnodeMap(126.55,36.76){rice1}\rput(rice1){\byeo} % 공주
\pnodeMap(127,36.35){rice2}\rput(rice2){\byeo\enskip\byeo} % 대전
\pnodeMap(126.65,35.35){rice3}\rput(rice3){\shortstack{\quad\byeo\\\qquad\byeo\\\quad\,\byeo\\\hspace{-.53em}\raisebox{.3em}{\byeo}}} % 전주
\pnodeMap(126.37,33.3){rice4}\rput(rice4){\byeo\hspace{-.7ex}\raisebox{.25ex}{\byeo}} %  제주
\pnodeMap(127.65,35.1){rice5}\rput(rice5){\shortstack{\enskip\byeo\\\byeo}} % 여수
\pnodeMap(128.38,35.22){rice6}\rput(rice6){\shortstack{\byeo\\[-.5ex]\enskip\byeo}} % 부산
\pnodeMap(128.7,35.8){rice7}\rput(rice7){\byeo\raisebox{2em}{\byeo}} % 대구
\pnodeMap(129.23,36.3){rice8}\rput(rice8){\byeo} % 포항
\pnodeMap(127.5,36.9){rice9}\rput(rice9){\byeo} % 원주
\pnodeMap(126.3,38.05){rice10}\rput(rice10){\byeo} % 개성
\pnodeMap(125.85,38.35){rice11} % 해주
\rput(rice11){{\shortstack{\enskip\,\byeo\hspace{1.2em}\raisebox{-.9em}{\byeo\raisebox{-.9em}{\byeo}}\\[-1.5em]
\hspace*{-3em}{\byeo}\raisebox{-.7em}{\quad\byeo}}}}

%%%  soy beans / 콩
\newcommand\kong{\scriptsize\textcolor{IndianRed}{\textbf{\Circpipe}}} % --> protosem.sty; textopenbullet
\pnodeMap(128.53,35.29){soy1}\rput(soy1){\kong} % 창원
\pnodeMap(128.9,36){soy2}\rput(soy2){\kong\qquad\raisebox{-.5em}{\kong\,\,\kong}} %  대구
\pnodeMap(126.7,36.45){soy3}\rput(soy3){\kong} %  공주
\pnodeMap(127.2,38.43){soy4}\rput(soy4){\kong\qquad\raisebox{-.3em}{\kong}} %  금강산
\pnodeMap(125.25,38.59){soy5}\rput(soy5){\kong} %  남포
\pnodeMap(125.8,38.4){soy6}\rput(soy6){\kong} %  사리원
\pnodeMap(125.2,38.15){soy7}\rput(soy7){\kong\qquad\raisebox{-1.8em}{\kong}}  %  해주
\pnodeMap(125.63,39.5){soy8}\rput(soy8){\kong\enskip\raisebox{-.8em}{\kong}}  %  안주
\pnodeMap(124.66,39.76){soy9}\rput(soy9){\kong} %  신의주
\pnodeMap(126.2,38){soy10}\rput(soy10){\kong\,\raisebox{-1.1em}{\kong}\quad\raisebox{-1em}{\kong}} %  개성

%%% tea / 차
\newcommand\cha{\scriptsize\textcolor{Magenta}{\textproto{\Asade}}} % --> protosem.sty
\pnodeMap(126.95,35.77){tea}\rput(tea){\shortstack{\cha\\\hspace*{-.9em}\cha\\[-.5em]\quad\cha}}  %  대전

%%% tobacco / 담배
\newcommand\dambae{\textcolor{Brown}{\textbf{\textleaf}}} % --> textcomp.sty
\pnodeMap(127.77,36.86){tobacco1}\rput(tobacco1){\large{\dambae}}  %  원주
\pnodeMap(127.06,36.28){tobacco2}\rput(tobacco2){\tiny{\dambae}}  %  대전

%%% mineral ressources / 지하 자원
\pnodeMap(127.2,37.37){gold}\psdot[dotstyle=BoldDiamond,fillcolor=Yellow,dotscale=3](gold) % 수원
\pnodeMap(128.5,40.95){copper}\psdot[dotstyle=BoldDiamond,fillcolor=red,dotscale=2.5](copper) % 혜산
\pnodeMap(125.45,38.25){iron1}\psdot[dotstyle=BoldDiamond,fillcolor=DodgerBlue,dotscale=2.5](iron1) %  해주
\pnodeMap(128.8,37.67){iron2}\psdot[dotstyle=BoldDiamond,fillcolor=DodgerBlue,dotscale=2](iron2) %  강능
\pnodeMap(126.7,36.7){lead1}\psdot[dotstyle=BoldDiamond,fillcolor=Green,dotscale=2.5](lead1) % 공주
\pnodeMap(128.7,40.6){lead2}\psdot[dotstyle=BoldDiamond,fillcolor=Green,dotscale=2.5](lead2) %  단천
\pnodeMap(129.4,41.35){bcoal}\psdot[dotstyle=BoldDiamond,fillcolor=Brown,dotscale=2.5](bcoal) % 청진
\pnodeMap(125.54,39.35){coal1}\psdot*[dotstyle=BoldDiamond,dotscale=2.5](coal1) %  평성
\pnodeMap(128.85,37.47){coal2}\psdot*[dotstyle=BoldDiamond,dotscale=2.5](coal2) %  강능
\pnodeMap(129.05,36.85){scoat1}\psdot[dotstyle=BoldDiamond,fillcolor=Coral,dotscale=3.8](scoat1) %  낙동강
\pnodeMap(127.4,38.6){scoat2}\psdot[dotstyle=BoldDiamond,fillcolor=Coral,dotscale=3](scoat2) % (포항)
\pnodeMap(127.2,39.87){scoat3}\psdot[dotstyle=BoldDiamond,fillcolor=Coral,dotscale=2.3](scoat3) % 함흥

%%% industrial production / 공업 생산
\pnodeMap(127.4,38){steel1}\rput(steel1){\textcolor{SteelBlue}{\Large\Industry}} % 서울
\pnodeMap(125.52,39.2){steel2}\rput(steel2){\textcolor{SteelBlue}{\Large\Industry}} % 평성
\pnodeMap(129.45,41.8){steel3}\rput(steel3){\textcolor{SteelBlue}{\Large\Industry}} % 청진
\pnodeMap(127.63,38.94){nonfer}\rput(nonfer){\textcolor{Coral}{\Large\Industry}} % 원산
\pnodeMap(127.75,40.1){alu1}\rput(alu1){\textcolor{Gray}{\Large\Industry}} %  함흥
\pnodeMap(125.39,38.92){alu2}\rput(alu2){\textcolor{Gray}{\Large\Industry}} % 평성
\pnodeMap(126.5,34.95){textile1}\psdot[dotstyle=Bo,fillcolor=YellowGreen,dotscale=2](textil1) % 목포
\pnodeMap(128.58,35.76){textile2}\psdot[dotstyle=Bo,fillcolor=YellowGreen,dotscale=2](textil2) %  대구
\pnodeMap(127,38){textil3}\psdot[dotstyle=Bo,fillcolor=YellowGreen,dotscale=2.3](textil3) % 서울
\pnodeMap(127.3,39.15){textil4}\psdot[dotstyle=Bo,fillcolor=YellowGreen,dotscale=2](textil4) %  원산
\pnodeMap(129.25,35.4){chem1}\psdot[dotstyle=Bo,fillcolor=Yellow,dotscale=2](chem1) %  울산
\pnodeMap(126.7,37.8){chem2}\psdot[dotstyle=Bo,fillcolor=Yellow,dotscale=2](chem2) %  서울
\pnodeMap(125.73,38.9){chem3}\psdot[dotstyle=Bo,fillcolor=Yellow,dotscale=2](chem3) %  평양
\pnodeMap(127.39,39.88){chem4}\psdot[dotstyle=Bo,fillcolor=Yellow,dotscale=2.5](chem4) %  함흥
\pnodeMap(130.1,42.2){chem5}\psdot[dotstyle=Bo,fillcolor=Yellow,dotscale=2.5](chem5) % 라선
\pnodeMap(126.73,36.2){metal1}\psdot[dotstyle=Bo,fillcolor=SteelBlue,dotscale=2.5](metal1) % 군산
\pnodeMap(126.95,37.36){metal2}\psdot[dotstyle=Bo,fillcolor=SteelBlue,dotscale=2.5](metal2) % 수원
\pnodeMap(129.3,36.7){metal3}\psdot[dotstyle=Bo,fillcolor=SteelBlue,dotscale=2.5](metal3) % (포항)
\pnodeMap(127.3,39){metal4}\psdot[dotstyle=Bo,fillcolor=SteelBlue,dotscale=2.5](metal4) % 원산
\pnodeMap(125.4,39.04){metal5}\psdot[dotstyle=Bo,fillcolor=SteelBlue,dotscale=2.5](metal5) %  평성
\pnodeMap(124.42,39.9){metal6}\psdot[dotstyle=Bo,fillcolor=SteelBlue,dotscale=2.5](metal6) %  신의주
\pnodeMap(127.33,39.72){metal7}\psdot[dotstyle=Bo,fillcolor=SteelBlue,dotscale=2.5](metal7) %  흥남
\pnodeMap(127.5,34.9){raff1}\psdot[dotstyle=BoldPentagon,fillcolor=BlueViolet,dotscale=2](raff1) % 여수
\pnodeMap(129.4,35.7){raff2}\psdot[dotstyle=BoldPentagon,fillcolor=BlueViolet,dotscale=3](raff2) % 울산
\pnodeMap(129.5,35.3){ship1}\rput(ship1){\large\textcolor{SteelBlue}{\anchor}} %  울산
\pnodeMap(126.4,37.4){ship2}\rput(ship2){\textcolor{SteelBlue}{\anchor}} % 인천
\pnodeMap(125.2,39.27){ship3}\rput(ship3){\textcolor{SteelBlue}{\anchor}} % 평성



%%%%%% layer 4 -- explanation / 설명 

%%%%%% korea2, layer3 -- legend and other additions / 설명 
%%%   kexpln.tex :: mito / nyamcoder


%%% railroad direction pointers / 철도 방향 지기
\pnodeMap(122.88,41.35){toDalian}
  \uput[0](toDalian){\sf\scriptsize\shortstack{다롄\\$\longleftarrow$}} % vs. \rput, pos. only directly at coor.
\pnodeMap(122.84,42.46){toTianjin}
  \uput[0](toTianjin){\sf\scriptsize\shortstack{북경\\톈진\\$\longleftarrow$}}
\pnodeMap(125.28,43.67){toChangchun}
  \uput[0]{-90}(toChangchun){\sf\scriptsize\shortstack{창춘\\$\longleftarrow$}}
\pnodeMap(129.33,43.54){toHarbin1}
  \uput[0]{-90}(toHarbin1){\sf\scriptsize{$\longleftarrow$}}
\pnodeMap(129.3,43.35){toHarbin2}
  \uput[0](toHarbin2){\sf\scriptsize{하얼빈, 블라디보스토크}}

% longitude--경도
\pnodeMap(125,43.74){t125}
  \rput(t125){$125^\circ$}
\pnodeMap(130,43.56){t130}
  \rput(t130){$130^\circ$}
\pnodeMap(125,31.97){b125}
  \rput(b125){$125^\circ$}
\pnodeMap(130,31.78){b130}
  \rput(b130){$130^\circ$}

% latitude--위도 
\pnodeMap(123.115,40){l40}
  \rput{-90}(l40){$\enskip 40^\circ$}
\pnodeMap(132.6,40){r40}
  \rput{90}(r40){$\enskip 40^\circ$}
\pnodeMap(125.105,35){l35}
  \rput{-90}(l35){$\enskip 35^\circ$}
\pnodeMap(132.1,35){r35}
  \rput{90}(r35){$\enskip 35^\circ$}

%%% outer frame / 곁틀
\psset{unit=1cm}	% cm를 필여 있음!; lgc218 
\psframe[framearc=.018,linewidth=.5pt](-3,-93)(11.5,-69.7)

%%% header / 지도 표제
\psset{linecolor=Tan,fillcolor=Tan,shadow=true,shadowcolor=Maroon,shadowsize=.05cm}
\bfseries\fontsize{30}{30}\selectfont	% emulated from 24.88pt
\rput(.2,-70.8){\pscharpath{한국전도}}

%%% caption / 지도 설명
\psset{linecolor=black} 
\psframe[fillcolor=Seashell,linewidth=.5pt](-2.85,-92.6)(-.13,-83)
\mdseries\fontsize{8}{8}\selectfont

% cultivation 
\rput(-1.49,-83.3){\it 곡물 재배}	% -.03 
\rput(-2.38,-83.7){\scalebox{.7}{\byeo\vspace*{-1em}\raisebox{.3em}{\byeo}}}
   \rput(-1.88,-83.7){벼}
\pscustom[linestyle=none]{\psframe(-2.13,-84.25)(-2.63,-83.95)
\fill[fillstyle=solid,fillcolor=LemonChiffon]
\fill[fillstyle=vlines,hatchcolor=Magenta,hatchwidth=.5pt]}
   \rput(-1.88,-84.1){밀}
\pscustom[linestyle=none]{\psframe(-2.63,-84.65)(-2.13,-84.35)
\fill[fillstyle=solid,fillcolor=Bisque]
\fill[fillstyle=hlines,hatchcolor=PaleGreen,hatchwidth=.5pt]}
   \rput(-1.755,-84.5){수수}
\rput(-1.1,-83.7){\kong\raisebox{.5em}{\kong}}
   \rput(-.7,-83.7){콩}
\rput(-1.1,-84.1){\cha\raisebox{.5em}{\cha}}
   \rput(-.73,-84.1){차}
\rput(-1.1,-84.5){\scalebox{.9}{\dambae}}
   \rput(-.58,-84.5){담배}

% mineral ressources 
\rput(-1.49,-85){\it 지하 자원}
\psdot*[dotstyle=BoldDiamond,dotscale=2](-2.38,-85.4)
   \rput(-1.88,-85.4){석탄}
\psdot[dotstyle=BoldDiamond,fillcolor=Brown,dotscale=2](-2.38,-85.8)
   \rput(-1.88,-85.8){갈탄}
\psdot[dotstyle=BoldDiamond,fillcolor=DodgerBlue,dotscale=2](-2.38,-86.2)
   \rput(-1.88,-86.2){철광}
\psdot[dotstyle=BoldDiamond,fillcolor=red,dotscale=2](-1.1,-85.4)
   \rput(-.58,-85.4){동광}
\psdot[dotstyle=BoldDiamond,fillcolor=Coral,dotscale=2](-1.1,-85.8)
   \rput(-.58,-85.8){정제}
\psdot[dotstyle=BoldDiamond,fillcolor=Yellow,dotscale=2](-1.1,-86.2)
   \rput(-.58,-86.2){금광}
\psdot[dotstyle=BoldDiamond,fillcolor=Green,dotscale=2](-2.38,-86.6)
   \rput(-1.64,-86.6){납\,·\,아연}

% industries 
\rput(-1.49,-87.1){\it 공업 생산}
\rput(-2.38,-87.5){\textcolor{SteelBlue}{\Industry}}
   \rput(-1.315,-87.5){철\,·\,강철 생산}
\rput(-2.38,-87.9){\textcolor{Coral}{\Industry}}
   \rput(-1.27,-87.9){비철금속 제련}  
\rput(-2.38,-88.3){\textcolor{Gray}{\Industry}}
   \rput(-1.27,-88.3){알루미늄 제철}
\psdot[dotstyle=Bo,fillcolor=SteelBlue,dotscale=2](-2.38,-88.8)
   \rput(-1.48,-88.7){금속 공업\,·}
   \rput(-1.55,-89){기계 제작}
\rput(-2.37,-89.4){\textcolor{SteelBlue}{\anchor}}
   \rput(-1.87,-89.4){조선}
\psdot[dotstyle=BoldPentagon,fillcolor=BlueViolet,dotscale=2](-2.38,-89.8)
   \rput(-1.6,-89.8){정유공장}
\psdot[dotstyle=Bo,fillcolor=Yellow,dotscale=2](-2.38,-90.2)
   \rput(-1.55,-90.2){화학 공업}
\psdot[dotstyle=Bo,fillcolor=YellowGreen,dotscale=2](-2.38,-90.6)
   \rput(-1.55,-90.6){섬유 산업}

% 일반
\psdot*[dotstyle=triangle](-2.38,-91.25)
   \rput(-2.05,-91.25){산}
\psset{shadow=false}
\psline[doubleline=true,linewidth=.5pt] (-1.63,-91.25)(-.98,-91.25) %  (-1.45,-91.25)(-.88,-91.25) 
\psline[linestyle=dashed,dash=4pt 4pt,linewidth=.5pt](-1.63,-91.25)(-.98,-91.25)
   \rput(-.61,-91.25){철도}
\psdot[dotstyle=triangle,linecolor=red,fillcolor=red](-2.38,-91.65)
   \rput(-1.9,-91.65){화산}
\fontsize{6}{6}\selectfont
\rput(-2.13,-92.05){\textcolor{DarkSeaGreen}{\sf\textbf{옌지(연길)}}}
  \rput(-.63,-92.05){중국}
  \rput(-1.51,-92.3){연변조선족자치주의\,도시}

%%% scale/축척
%\psset{shadow=false}
\rput(9.7,-73.1){축척 $1:6\,000\,000$}
   \psline[doubleline=true,linewidth=.7pt](8.5,-73.5)(10.9,-73.5)
   \psline[linestyle=dashed,dash=8mm 8mm,linewidth=.8pt](8.5,-73.5)(10.9,-73.5)
\fontsize{5}{5}\selectfont
   \rput(8.5,-73.35){0}
   \rput(9.3,-73.35){50}
   \rput(10.1,-73.35){100}
   \rput(10.9,-73.35){150}
   \rput(11.25,-73.35){km}
\fontsize{6}{6}\selectfont
\rput(9.7,-73.9){\shortstack{지도의 1cm는 실제의\\[-.3em] 60km와 일치합니다.}}



%%%%%% helping grid / 보조 눈금 
%\psgrid[linewidth=.7pt,subgriddiv=2,gridcolor=red,gridlabelcolor=red](4,-80)(-3,-93)(11.5,-69)


\end{pspicture*}%} % --> \scalebox
\end{document} 


* compilation

$ latex korea2 && dvips korea2 && gs -q -dBATCH -sDEVICE=pdfwrite "-sOutputFile=korea2.pdf" -c .setpdfwrite -f "korea2.ps"
<return>
>>showpage, press <return> to continue<<
>>showpage, press <return> to continue<<

or 3x <return> at once
