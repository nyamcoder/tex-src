\documentclass{article}  %%% ktree.tex :: mito / nyamcoder %%%
\usepackage{CJKutf8} %  또는 hangul/kotex 을 상용
\usepackage{pstricks,pst-xkey,pst-jtree}
\pagestyle{empty}

\begin{document}
\begin{CJK}{UTF8}{mj}
\vspace*{-3cm}

\begin{center}
\jtree[xunit=2em,yunit=1.5em]
\! = {즉} ({-나}[labelgap=.7em]) :[scaleby=3]!a !c .
\!a = :!b {가능하-,}.
\!b = :{학습으로} :{수행능력의\qquad} {향상이}.
\!c = :{문학은} :!d [scaleby=4]:{문학적,} :{ 예술적이어야 \quad} :{하는} {것이다.}.
\!d = :!e [scaleby=2]:{재표현의\qquad\quad} {글쓰기가}.
\!e = :{예술의\quad} {영역이므로}.
\endjtree\\[1em]
\end{center}

\noindent  즉 학습으로 수행능력의 향상이 가능하나,  문학은 예술의 영역아므로 재표현의 글쓰기가 문학적, 예술적이어야 하는 것이다.\\[3em]
\small
%\begin{center}
\hspace*{-3.35em}
\jtree[xunit=2em,yunit=1.5em]
\! = {Das heißt, dass} ({-- aber}[labelgap=.85em]) :[scaleby=4]!a !d .
\!a = :[scaleby=3]!b !c .
\!b = :{es} :{zwar} :{möglich} {ist,}.
\!c = :{das Niveau\qquad} :{\shortstack{übersetzerischer\\Fertigkeiten}\qquad\qquad} :{\shortstack{durch\\Einübung}} {\qquad\quad zu erhöhen --}.
\!d = :!g [scaleby=4]:{\shortstack{sollte sich\\das}} [scaleby=5]:!f [scaleby=4]:{widerspiegeln\qquad} :[scaleby=2]!h {\qquad\,\textbf{.}}.
\!g = :{da} :{die Literatur\qquad} :{\quad\shortstack{eine\\Kunstform}} {ist,}.
\!f = :{in den\qquad} :{\shortstack{schriftlichen\\[-.3em]Übertragungen der}\qquad\qquad}{\qquad\quad Ausdrücke}@X .
\!h = :{, die\quad}@Y :{von} :[scaleby=2]!i :{Wert\quad}{\quad sein müssen(,)}.
\!i = :{\shortstack{literarischem\\\quad}\qquad} {\shortstack{und\\\qquad\qquad künstlerischem}}.
\nccurve[angleA=235,angleB=260,linestyle=dashed]{->}{Y}{X}
\endjtree\\[1em]
%\end{center}

\normalsize\noindent
Das heißt, dass es zwar möglich ist, das Niveau übersetzerischer Fertigkeiten durch Einübung zu erhöhen -- aber da die Literatur eine Kunstform ist, sollte sich das in den schriftlichen Übertragungen der Ausdrücke widerspiegeln, die von literarischem und künstlerischem Wert sein müssen.

\end{CJK}
\end{document}

* compilation
$ latex ktree.tex
$ dvips ktree.dvi
$ ps2pdf ktree.ps

* documentation
/usr/share/texlive/2008/texmf-dist/doc/generic/pst-jtree/pst-jtree-doc.pdf 

* KO+EN souce text from:
M. ­K. Choi/최미경,한국문학번역고육: 이론과 실제, in: 1st International Translators’ Conference – Korean Culture in Europe: Achievements and Prospects, Korea Literature Translation Institute (ed.), Seoul 2007, vol. II, p507f/p518
