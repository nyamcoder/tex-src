%%% toc-de.tex :: mito / nyamcoder %%%
\documentclass[a4paper,11pt]{scrartcl}% ,ngerman
\usepackage{boxedminipage}% mit dem paket keine kor.sg 
\usepackage[ngerman]{babel}
\usepackage{kotex} % bleibt dann kor, auch mit ngerman/babel		[hanja]
\usepackage{times}
\setlength\textheight{27.5cm}
\setlength\voffset{-2cm}
\setlength\textwidth{16cm}

\begin{document}	% toc: 48ff 
\begin{boxedminipage}[t]{\linewidth}
\tableofcontents
\dotfill
\section*{Vorwort}
\addcontentsline{toc}{section}{\protect\numberline{}Vorwort}
Koreas umfangreiches Kulturvermächtnis enthält viele als Nationalschätze klassifizierte Güter\,\cite{k1}, die darüber hinaus als Welterbe in die UNESCO-Liste aufgenommen wurden.\cite{k2} 
\section{Die Nationalschätze Südkoreas (\textit{kukpo})}
\subsection{Orte und Gebäude}
\subsubsection{Namdaemun (1395/1447), Nr. 1}
\subsubsection{Der Pulguksa-Tempelkomplex (774), UNESCO-Weltkulturerbe}
\paragraph{Tabot'ap-Pagode (751), Nr. 20} 		% nicht mehr gezählt, im toc nicht angezeigt 
\paragraph{S\u{o}kkat'ap-Pagode (751), Nr. 21} 	% nicht mehr gezählt, im toc nicht angezeigt 
\subsubsection{S\u{o}kkuram-Grotte (751--774), Nr. 24, UNESCO-Weltkultur- und Naturerbe}
\subsection{Objekte der Bildenden Kunst und des Kunsthandwerks}
\subsubsection{Vergoldete Bronzefigur des Mir\u{u}k (7. Jh.), Nr. 83}
\subsection{Gemälde, Schriftstücke und Druckmedien}
\subsubsection{Tripitaka Koreana (1236--1251), Nr. 32, UNESCO-Weltdokumentenerbe}
\subsubsection{Hunminj\u{o}ng\u{u}m (1443), Nr. 70, UNESCO-Weltdokumentenerbe}
%\subsubsection{Cheonmado (5./6. Jh.), Nr. 207}
%\section{Die Schätze Koreas (\textit{pomul})}
%\subsubsection{} ...
%\subsubsection{}
%\section{Die Nationalschätze Nordkoreas (\textit{kukpo})}
%\subsubsection{} ...
\section{Überregionales Kulturerbe}
\subsection{Der Kogury\u{o}-Gräberkomplex, UNESCO-Weltkulturerbe \textnormal{\cite{k3}}}

\renewcommand\refname{Weblinks}
\begin{thebibliography}{9}
\addcontentsline{toc}{section}{\refname}
\bibitem{k1} \verb#http://www.heritage.go.kr#, \enskip\verb#http://www.cha.go.kr#
\bibitem{k2} \verb#http://german.visitkorea.or.kr/ger/CU/CU_GE_5_10_0.jsp#
\bibitem{k3} \verb#http://whc.unesco.org/en/list/1091#
%\bibitem{k4} \verb#http://www.eu-asien.de/Korea-Informationen/Touristeninformationen/UNESCO-2009-Korea.html# % <-- allg. 
\end{thebibliography}

\end{boxedminipage}
\end{document}


~~~~~~~~~~~~~~~~~~~~~~~~~~~~~~~~~~~~~~~~~~~~~~~~~~~~~~~~~~~~~~~~~~~~~~~~~~~


* kompilation

$ pdflatex toc-de.tex && pdflatex toc-de.tex


--> erzeugt beim ersten durchlauf toc-de.toc, die beim zweiten verwendet wird

